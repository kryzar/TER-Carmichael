\section*{Introduction}
\addcontentsline{toc}{section}{Introduction}

\begin{question}\label{question-centrale}Soient $n$ un entier de Carmichael et $K$ un corps de nombre. Dans quel mesure $n$ est-il de Carmichael dans $K$ ?
\end{question}

\begin{equation}\label{eq-Carmichael}
\alpha^{\Norm(I)} \equiv \alpha \pmod{I}, \quad \forall \alpha \in \OK.
\end{equation}

Nous étudions tous les nombres de Carmichael de la liste\footnote{La liste des premiers entiers de Carmichael est disponible à \url{https://oeis.org/A002997}.} 
\begin{align*}
	\mathfrak{C} = \{& 561, 1105, 1729, 2465, 2821, 6601, 8911, 10585, 15841, 29341, 41041,\\ 
	& 46657, 52633, 62745, 63973, 75361, 101101, 115921, 126217, 162401, \\ 
	& 172081, 188461, 252601, 278545, 294409, 314821, 334153, 340561, \\
	& 399001, 410041, 449065, 488881, 512461\}
\end{align*}

