\section{Corps quadratiques}

\subsection{Vers un possible test de primalité}

Entrons dès à présent dans le vif du sujet. L'un de nos objectifs principaux de répondre à la question \ref{question-centrale}. Le théorème 2.5 de l'article y apporte de premiers éléments de réponse.

\begin{theoreme}[2.5 dans l'article]\label{theoreme-2.5}
	Soit $n$ un entier impair sans facteurs carrés. S'il existe un diviseur premier $p$ de $n$ tel que $$p^2 - 1 \nmid n^2 - 1,$$ alors il existe une infinité de corps quadratiques $K$ dans lesquels $n$ \emph{n'est pas} de Carmichael.
\end{theoreme}

\begin{MotSurPreuve}
	On présage dès l'énoncé la nature de la preuve : c'est le critère de Korselt généralisé \ref{korselt-generalise}. On commence par s'assurer que $n$ est sans facteurs carrés de sorte de contrôler la décomposition de $n\OK$ dans un corps quadratique $K$ donné. Du reste, comme la norme d'un idéal premier au dessus de $n\OK$ est un nombre premier $p$ divisant $n$, on sent bien que la condition de non-divisibilité va empêcher être $n$ d'être de Carmichael dans les corps quadratiques\footnote{Les carrés ne sont ni plus ni moins que les normes de $p$ et de $n$ dans toute extension quadratique.} bien choisis. La partie réellement inventive de la preuve consiste à trouver les bons corps quadratiques. Cette partie est hautement non triviale et est basée sur la connaissance d'un nombre premier $p$ comme dans les hypothèses du théorème et sur une savante utilisation du théorème chinois. Les techniques de base de la ramification permettent encore une fois de s'assurer que les corps construits vérifient bien ce qu'on leur demande de vérifier. \\
\end{MotSurPreuve}

Bien que cet énoncé ne semble pas optimal en pratique\footnote{Il n'y a à ce jour (3 juin 2020) pas d'algorithme efficace pour déterminer si un entier est sans facteurs carrés. Les algorithmes passent souvent par la décomposition en produit de facteurs premiers, ce qui ne nous arrange pas. \NTS{sourcer}.}, certains nombres de Carmichael vérifient ces hypothèses. C'est le cas par exemple du nombre de $512461$ et de son facteur premier $271$. Il existe ainsi une infinité de corps quadratiques dans lequel $512461$ n'est pas de Carmichael. Mieux encore, nous avons numériquement exhibé pour chaque entier de Carmichael de la liste \ref{liste-carmichael} une liste de corps quadratiques dans lequel ledit entier n'est pas de Carmichael. \NTS{ajouter liste} \\

\begin{Complement}
	Une liste bien plus dense exhibée par l'auteur de ce mémoire est disponible sur sa page GitHub : \url{https://github.com/kryzar/ter-carmichael/blob/master/scripts/results_carmichael_not_carmichael_in_quad_field.txt}. \\
\end{Complement}

Malgré cela, l'algorithme que nous avons utilisé pour trouver ces corps est sans doute très sous-optimal. Utilisant le critère de Korselt généralisé \ref{korselt-generalise}, il impose de décomposer l'idéal $n\OK$ en produit d'idéaux premiers\footnote{\NTS{Si possible, donner un mot sur la complexité de l'algo de décomposition dans un extension}}. Dans l'exemple 2.6 de l'article, l'auteur ouvre une voix bien plus intéressante. Il montre que $n = 561$ est composé en exhibant le corps quadratique $K = Q(\sqrt{13})$ puis l'entier algébrique $\alpha = 2 + 1\cdot \left(\frac{1 + \sqrt{13}}{2}\right) \in \OK$. Comme $$\alpha^{\Norm_{K/\Q}(n)} \not \equiv \alpha \pmod{n\OK}$$ et que $n$ et $13$ sont premiers entre eux, cela prouve la composition de $n$. Le point clé est que l'auteur ne semble pas utiliser le critère de Korselt (il n'a pas donné les détails et nous ne pouvons en conséquence pas en être sûrs). Cette approche nous invite à étudier l'algorithme \emph{probabiliste} suivant. Soit $n$ un entier impair (potentiellement de Carmichael).

\vspace{1em}
\begin{algorithm}[H]\label{algo-quadratique-faible}
\KwIn{$b$ (borne coordonnées $\alpha$)}
\ForEach{$d \in \N$ sans facteurs carrés}{
	$K \leftarrow \Q(\sqrt{d})$ \\
	$\theta\leftarrow$ un générateur de $\OK$ \\
	\If{$d$ et $n$ sont premiers entre eux}{
		\ForEach{$\alpha = x+ y\theta$, $x, y\in [\![1, b]\!]$}{
			\If{$\alpha^{n^2}\not\equiv \alpha \pmod{n\OK}$}{
				\KwRet{$n$ n'est pas de Carmichael dans $\Q(\sqrt{d})$ et est composé}	\\
				\textbf{fin du programme}
			}
		}
	}
}
\end{algorithm}
\vspace{1em}

Pour que cet algorithme soit viable en pratique, il conviendra de définir la notion de \textit{témoin de Carmichael} et d'étudier finement leur répartition, à la manière du test de primalité de Rabin-Miller (voir § 2.3.7, p.69 de \cite{Demazure}). L'auteur de ce présent mémoire est pleinement conscient du travail à accomplir. Par ailleurs, il n'y aucune preuve que l'algorithme termine en l'état. \\

Il peut en outre être tentant d'utiliser cet algorithme pour déterminer la primalité de $n$ avec une certitude morale, toujours dans l'esprit test de primalité de Rabin-Miller. L'idée serait que si $n$ vérifie la congruence $\alpha^{\Norm_{\Q(\sqrt{d})/\Q}} \equiv \alpha \pmod{n\OK}$ pour un nombre suffisamment grand de corps quadratiques $K = \Q(\sqrt{d})$ de discriminant premier avec $n$ et d'entiers algébriques $\alpha \in \OK$, nous aurions une certitude morale de la primalité de $n$. Mais l'existence d'entiers $h$ qui sont de Carmichael \emph{dans tout corps quadratique de discriminant premier avec $h$} est un problème majeur qui empêche de considérer cette voie. C'est le propos de la prochaine sous-section.

\subsection{Un pas en arrière}

Revenons un pas en arrière. Nous avons évoqué la réciproque du petit théorème de Fermat (\ref{ptf-reciproque}). Il faut bien faire attention au fait que l'énoncé suivant est faux~!

\begin{enoncefaux}[réciproque eronnée du petit théorème de Fermat généralisé]
	Soit $n>2$ un entier. Alors $n$ est premier \ssi pour tout corps \emph{quadratique} $K$ vérifiant $\pgcd(n, \Disc(K)) = 1$ et tout entier algébrique $\alpha \in \OK$, on a $$\alpha^{\Norm_{K/\Q}(n)} \equiv \alpha \pmod{n\OK}.$$
\end{enoncefaux}

Si l'on veut prouver qu'un entier $n>2$ est premier, montrer que la congruence précédente est vérifiée pour tous les corps quadratiques de discriminant premier avec $n$ ne suffit pas. Il faut s'assurer pour tous les corps quadratiques de discriminant non divisé par $n$. Même si ces notions sont les mêmes lorsque $n$ est effectivement premier, cette condition est beaucoup plus forte~! Il est même facile de démontrer le résultat suivant.

\begin{theoreme}[2.7 dans l'article]\label{theoreme-2.7}
	Soient $n$ un entier composé sans facteur carré et $d\geq 1$ un entier. Si pour tout diviseur $p$ de $n$ et tout enter $1\leq i \leq d$ la division $$p^i - 1\mid n^d - 1$$ est vérifiée, alors $n$ est de Carmichael dans $\Q$ et tout corps de nombres de degré $d$ de discriminant premier avec $n$.
\end{theoreme}

\begin{definition}
	Un entier $n$ vérifiant les hypothèses du théorème \ref{theoreme-2.7} pour un certain entier $d\geq 1$ est appelé \emph{nombre de Carmichael rigide d'ordre $d$}.
\end{definition}

Chose troublante, de tels nombres existent.

\begin{theoreme}[Howe, 2000]
	Il existe un couple $(n, d)$ d'entiers tel que $n$ soit un entier de Carmichael rigide d'ordre $d$.
\end{theoreme}

En guise de preuve, E. W. Howe exhibe un entier de Carmichael rigide d'ordre $2$. Il est donné par 
	\begin{equation}\label{Howe}
		h = 17 \cdot 31 \cdot 41 \cdot 43 \cdot 89 \cdot 97 \cdot 167 \cdot 331 = 443372888629441
	\end{equation}
et est appelé \emph{entier de Howe}. \\

À la lumière de ces troublants résultats, si l'on veut pouvoir trouver un test de primalité \emph{alla} Rabin-Miller comme évoqué dans la sous-section précédente, il faudra modifier l'algorithme (\ref{algo-quadratique-faible}) en la version suivante.


\vspace{1em}
\begin{algorithm}[H]\label{algo-quadratique-faible}
\KwIn{$I$ (intervalle d'entiers $d$ sans facteurs carrés), $b$ (borne coordonnées $\alpha$)}
\ForEach{$d \in \N$ sans facteurs carrés dans $I$}{
	$K \leftarrow \Q(\sqrt{d})$ \\
	$\theta\leftarrow$ un générateur de $\OK$ \\
	\If{$n$ et $\Disc(K)$ sont premiers entre eux}{
		\ForEach{$\alpha = x+ y\theta$, $x, y\in [\![1, b]\!]$}{
			\If{$\alpha^{n^2}\not\equiv \alpha \pmod{n\OK}$}{
				\KwRet{$n$ n'est pas de Carmichael dans $\Q(\sqrt{d})$ et est composé}	\\
				\textbf{fin du programme}
			}
		}
	}
}
\KwRet{$n$ a une certitude morale d'être premier}
\end{algorithm}
\vspace{1em}

Il y a de nouveau un travail substantiel à faire pour s'assurer de la viabilité de cette direction. Il ne faut pas non plus négliger d'autres pistes, c'est pourquoi nous partons explorer le royaume des corps cyclotomiques.
