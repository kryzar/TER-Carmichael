\section{Interlude : idéaux de Carmichael et extensions de corps}

Donnons nous \[\Q\subset K \subset L\] une tour de corps de nombres, $I\subset \OK$ un idéal de Carmichael de $K$ et $J \subset \OL$ un idéal de Carmichael de $L$. 

\begin{question}
	L'idéal étendu $I\OL$ est-il de Carmichael dans $L$ et l'idéal restreint $J\cap\OK$ est-il de Carmichael dans $K$ ? 
\end{question}

L'auteur de l'article a déjà répondu à la première partie de la question dans l'exemple 2.6 de l'article, en utilisant le test de Fermat dans les corps quadratiques.

\begin{proposition}\label{premier-ce}
	Il existe des tours de corps de nombres $\Q\subset K \subset L$ et des idéaux de Carmichael $I$ de $K$ pour lesquels l'idéal étendu $I\OL$ qui ne sont pas de Carmichael dans $L$.
\end{proposition}

\begin{proof}
	L'exemple 2.6 de l'article montre que l'entier de Carmichael $561$ n'est pas de Carmichael dans $\Q(\sqrt{13})$. On a alors l'énoncé en prenant $K = \Q$, $I = 561\OK$ et $L = \Q(\sqrt{13})$.
\end{proof}

L'algorithme de Korselt \ref{test-primalite-korselt} dans les corps quadratiques permet de répondre à la deuxième partie de la question. 

\begin{proposition}\label{deuxieme-ce}
	Il existe des tours de corps de nombres $\Q\subset K \subset L$ et des idéaux de Carmichael $J$ de $L$ pour lesquels l'idéal restreint $J\cap \OK$ n'est pas de Carmichael dans $K$.
\end{proposition}

Pour y répondre, l'auteur du présent mémoire a implémenté l'algorithme Suivant.

\vspace{1em}
\begin{algorithm}[H]\label{test-Fermat-quadratique}
\caption{trouver un entier rationnel $n$ et un corps quadratique $K$ tels que $n$ ne soit pas de Carmichael dans $K$}
\KwIn{$\mathscr{N}$ (liste d'entiers qui ne sont \emph{pas} de Carmichael), $\mathscr{K}$ (liste de corps quadratiques)}
\KwOut{couples $(n, K)$ où $K$ est un corps quadratique de $\mathscr{K}$ et $n$ est est un entier rationnel de $\mathscr{N}$ de Carmichael dans $K$ mais pas dans $\Q$}
\ForEach{$n$ dans $\mathscr{N}$}{
	\ForEach{$K$ dans $\mathscr{K}$}{
		\If{$n$ est de Carmichael dans $K$ (critère de Korselt)}{
			\KwRet{$n$ n'est pas de Carmichael mais $n\OK$ l'est}	\\
			\textbf{arrêter le programme}
		}
	}
}
\end{algorithm}
\vspace{1em}

Cette algorithme retourne de nombreux résultats. Par exemple $35$ est de Carmichael dans $\Q(\sqrt{11})$ et $8029$ est de Carmichael dans $\Q(\sqrt{-73})$ ~; aucun de ces deux nombres n'est de Carmichael dans $\Q$. \NTS{lien vers résultats complets} Nous avons tiré les statistiques suivantes.

\begin{table}[H]
	\begin{center}
		\begin{tabular}{|c|c|}
			\hline
			donnée évaluée & statistique \\
			\hline
			\hline
			nombre de couples testés & 5723 \\\hline
			nombre d'entiers qui sont de Carmichael dans un corps testé & 2930 \\\hline
			proportion d'entiers qui sont de Carmichael dans un corps testé & 51,2 \% \\\hline
		\end{tabular}
		\caption{Statistiques des simulations du critère de Korselt sur les corps quadratiques pour les paramètres \ref{param-korselt-quadra}.}
	\end{center}
\end{table}

\begin{remarque}
L'entier $n = 8029$ a la particularité d'être le produit de trois nombres premiers distincts : \[n = 8029 = 7 \cdot 31 \cdot 37.\] Le fait qu'il existe un corps quadratique dans lequel $n$ n'est pas de Carmichael peut se traduire dans le cadre de l'arithmétique classique grâce au critère de Korselt : existe-t-il un tripler de nombres premiers distincts $p$, $q$ et $r$ pour lesquels \[(pqr)^2 - 1\] est divisible à la fois par $p-1$, $q-1$ et $r-1$ ? Les diviseurs premiers de $8029$ sont un tel triplet.
\end{remarque}

Face à la généralité de l'énoncé de la question posée, il était tentant d'aller chercher une réponse ou bien théorique, ou bien dans des corps de nombres beaucoup plus compliqués. En fin de compte, les corps quadratiques auront suffit. Nous avons vu à la section précédente que sur les entiers de Carmichael et les corps quadratiques testés, ceux-ci restaient de Carmichael dans environ 24 \% des cas. Ici, nous avons montré que sur les entiers n'étant pas de Carmichael et les corps quadratiques testés, ceux-ci engendraient un entier de Carmichael dans plus de 50 \% des cas. Ces statistiques sont symptomatiques du fait que nous sommes a priori largement ignorants sur le comportement d'un idéal de Carmichael lorsqu'on l'étend ou le restreint, et ce même en restant dans le cadre des corps quadratiques, corps de nombres a priori parmi les moins compliqués.
