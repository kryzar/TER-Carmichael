\section{Interlude : idéaux de Carmichael et extensions de corps}

Soit \[\Q\subset K \subset L\] une tour de corps de nombres, $I\subset \OK$ un idéal de Carmichael de $K$ et $J \subset \OL$ un idéal de Carmichael de $L$. L'idéal étendu $I\OL$ est-il de Carmichael dans $L$ et l'idéal restreint $J\cap\OK$ est-il de Carmichael dans $K$ ? \\

L'auteur de l'article a déjà répondu à la première partie de la question dans l'exemple 2.6 de l'article, en utilisant le test de Fermat dans les corps quadratiques.

\begin{proposition}\label{premier-ce}
	Il existe des tours de corps de nombres $\Q\subset K \subset L$ et des idéaux de Carmichael $I$ de $K$ pour lesquels l'idéal étendu $I\OL$ qui ne sont pas de Carmichael dans $L$.
\end{proposition}

\begin{proof}
	L'exemple 2.6 de l'article montre que l'entier de Carmichael $561$ n'est pas de Carmichael dans $\Q(\sqrt{13})$. On a alors l'énoncé en prenant $K = \Q$, $I = 561\OK$ et $L = \Q(\sqrt{13})$.
\end{proof}

L'auteur du présent mémoire répond à la deuxième partie de la question en implémentant le critère de Korselt \ref{test-primalite-korselt} dans les corps quadratiques et donnant des exemples de tels idéaux corps de nombres.

\begin{proposition}\label{deuxieme-ce}
	Il existe des tours de corps de nombres $\Q\subset K \subset L$ et des idéaux de Carmichael $J$ de $L$ pour lesquels l'idéal restreint $J\cap \OK$ n'est pas de Carmichael dans $K$.
\end{proposition}

\begin{proof}
Donnons plusieurs exemples. On peut prendre :
\begin{itemize}
	\item $K = \Q$, $L = \Q(\sqrt{11})$ et $J = 35\OL$~;
	\item $K = \Q$, $L = \Q(\sqrt{95})$ et $J = 8029\OL$.
\end{itemize}
\end{proof}

\NTS{lien vers résultats complets}

\begin{remarque}
L'entier $n = 8029$ a la particularité d'être le produit de trois nombres premiers distincts : \[n = 8029 = 7 \cdot 31 \cdot 37.\] Le fait qu'il existe un corps quadratique dans lequel $n$ n'est pas de Carmichael peut se traduire dans le cadre de l'arithmétique classique grâce au critère de Korselt : existe-t-il un tripler de nombres premiers distincts $p$, $q$ et $r$ pour lesquels \[(pqr)^2 - 1\] est divisible à la fois par $p-1$, $q-1$ et $r-1$ ? Les diviseurs premiers de $8029$ sont un tel triplet.
\end{remarque}

Face à la généralité de l'énoncé de la question posée, il était tentant d'aller chercher une réponse ou bien théorique, ou bien dans des corps de nombres beaucoup plus compliqués. Le fait d'avoir trouvé une réponse avec deux algorithmes naïfs (test de Fermat et critère de Korselt) est symptomatique du fait que nous sommes largement ignorants sur le comportement d'un idéal de Carmichael lorsqu'on le restreint ou l'étend, et ce même dans le cas de corps aussi simples que les corps quadratiques. Nous verrons dès la prochaine sous-section de nombreux exemples de nombres de Carmichael qui le restent ou ne le restent pas dans tels ou tels corps quadratiques puis dans tels ou tels corps cyclotomiques.
