\section{Corps cyclotomiques}

\subsection{Théorie}

Dans cette section, nous nous intéressons aux corps cyclotomiques de la forme $\Q(\zeta_q)$ où $q$ est un nombre premier. Commençons par un résultat théorique.

\begin{theoreme}[3.1 dans l'article]
	Pour tout entier naturel $n$ composé, il existe une infinité de corps de nombres \emph{abéliens} de discriminant premier avec $n$ dans lesquels $n$ \emph{n'est pas de Carmichael}.
\end{theoreme}

\begin{MotSurPreuve}
	Les preuves des énoncés sur les corps cyclotomiques sont autrement plus sophistiquées que celles sur les corps quadratiques et l'auteur y utilise à volonté des résultats de théorie analytique des nombres. Ces derniers assurent de l'existence d'objets mais ne les construisent pas, à l'image des lemmes 3.3 et 3.4 de l'article. Pour ce théorème, la construction des corps se fait avec la correspondance de Galois et la vérification et la vérification qu'ils vérifient les bonnes propriétés nécessite des arguments sophistiqués de ramification (groupes de décomposition et d'inertie, Frobenius d'un élément). Nous verrons ces arguments plus en détails dans la preuve du théorème 3.6 de l'article (n°\ref{theoreme-3.6} chez nous). \\
\end{MotSurPreuve}

Un nombre de Carmichael étant composé, il vérifie les hypothèses du théorème. Cela fournit une nouvelle réciproque au petit théorème de Fermat, plus contraignante que la précédente.

\begin{theoreme}[deuxième réciproque]
	Soit $n$ un entier. Alors $n$ est premier \ssi pour tout corps de nombres \emph{abélien} $K$ de discriminant premier avec $n$, $n$ n'admet aucun $K$-témoin de Fermat.
\end{theoreme}

Le résultat le plus à même d'aboutir à un test de primalité est le crucial théorème suivant. Il nous enseigne qu'il faut aller chercher du côté des corps cyclotomiques. Nous en donnons la démonstration.

\begin{theoreme}[3.6 dans l'article]\label{theoreme-3.6}
	Soit $n$ un entier composé ayant au moins trois facteurs premiers distincts. Alors il existe une infinité de corps cyclotomiques $K$ de la forme $K = \Q(\zeta_q)$, $q$ étant premier, tels que $\Disc(K)$ est premier avec $n$ et $n$ \emph{n'est pas de Carmichael dans $K$}.
\end{theoreme}

\begin{proof}
	\NTS{placeholder}
\end{proof}

Un nombre de Carmichael ayant toujours au moins trois diviseurs premiers distincts (voir \cite{Demazure}, Proposition 3.35, p. 90), il vérifiera toujours les hypothèses du théorème. Ce résultat est donc en théorie bien plus puissant que le théorème \ref{theoreme-2.5} (2.5 dans l'article), puisque nous avons vu que le test de primalité naïf qui en découlait ne détectait pas tous les entiers de Carmichael, comme par exemple l'entier de Howe \ref{Howe}. 

\begin{corollaire}[3.7 dans l'article]\label{corollaire-3.7}
	Soit $n$ un entier composé. Il existe au moins un corps cyclotomique de la forme $\Q(\zeta_q)$, $q$ étant premier, de discriminant premier avec $n$ dans lequel $n$ n'est pas de Carmichael.
\end{corollaire}

\begin{proof}
	Distinguons trois cas.
	\begin{itemize}
		\item Si $n$ a un unique facteur premier, étant composé, il a un facteur carré et ne sera jamais de Carmichael, d'après le critère de Korselt généralisé \ref{korselt-generalise}.
		\item Si $n$ a deux uniques facteurs premiers distincts, il n'est pas de Carmichael dans $\Q$, le corps cyclotomique $\Q(\zeta_2)$ (voir \cite{Demazure}, Proposition 3.35, p. 90).
		\item Si $n$ a au moins trois facteurs premiers distincts, nous appliquons le théorème précédent.
	\end{itemize}
\end{proof}

Ce corollaire a bien entendu droit à sa réciproque du théorème de Fermat.

\begin{theoreme}[troisième réciproque]
	Soit $n$ un entier. Alors $n$ est premier \ssi pour tout corps cyclotomique $K$ de la forme $K = \Q(\zeta_q)$, $q$ étant premier, $n$ n'admet aucun $K$-témoin de Fermat.
\end{theoreme}

\begin{remarque}
	Dans les deux réciproques que nous venons de donner, l'hypothèse sur le discriminants est que $n$ et $K$ doivent être premiers entre eux. C'est une hypothèse bien plus forte que de demander à ce que $n$ ne divise pas le discriminant de $K$ comme dans les corps quadratiques (\ref{ptf-reciproque})et cela permet de tester beaucoup moins de corps.
\end{remarque}

Passons désormais aux simulations numériques.

\subsection{Simulation}

\subsubsection{Test de Fermat}

Étant donné un entier de Carmichael $n$, la version \og cyclotomique \fg{} du test de Fermat serait la suivante.

\vspace{1em}
\begin{algorithm}[H]\label{test-Fermat-cyclotomique}
\caption{Test de Fermat dans un corps cyclotomique}
\KwIn{$n$ (entier à tester), $K$ (corps cyclotomique de la forme $\Q(\zeta_q)$ où $q$ est premier), $S_\alpha$ (ensemble coordonnées $\alpha$)}
\If{$n$ et $\Disc(K)$ sont premiers entre eux}{
	$\zeta \leftarrow$ une racine primitive $q$-ième de l'unité engendrant $K$, $q$ étant premier \\
	\ForEach{$\alpha = x_0 + x_1\zeta + \dots + x_{p-1}\zeta^{p-1}$, $x_0, \dots, x_{p-1}\in S_\alpha$}{
		\If{$\alpha^{n^{p-1}}\not\equiv \alpha \pmod{n\OK}$}{
			\KwRet{$n$ n'est pas de Carmichael dans $K$ et est composé} \\
			\textbf{arrêter le programme}
		}

	}
}
\end{algorithm}
\vspace{1em}

Cet algorithme s'est montré encore moins performant que sa version quadratique. En posant $n=561$, $K = \Q(\zeta_5)$ et $S_\alpha = \{-1, 0, +1\}$, l'algorithme n'a toujours pas terminé après avoir tourné pendant une heure sur l'ordinateur personnel de l'auteur. Ce n'est finalement pas si étonnant. On aura \[\Norm(n\OK) = 561^4 = 99049307841\] et si l'on prend par exemple l'entier algébrique de coordonnées $(1, 1, 0, 0)$ \[\alpha =1 + \zeta,\] l'ordinateur portable de l'auteur n'a toujours pas su calculer $\alpha^{\Norm(n\OK)}$ après trente minutes de calculs\footnote{Wolfram alpha n'a pas fait mieux.}. Les coordonnées ont pourtant été choisies pour être les plus petites possibles, mais le coût du calcul de $\alpha^{\Norm(n\OK)}$ est prohibitif. Plus généralement, le calcul de $\alpha^{\Norm(n\OK)} = \alpha^{n^{q-1}}$ sort rapidement des capacités d'un ordinateur personnel standard.

\subsubsection{Critère de Korselt}

Comme dans la section sur les corps quadratiques, nous cherchons ici — pour tout entier $n$ de la liste \ref{liste-carmichael} — un corps cyclotomique de la forme $\Q(\zeta_q)$, $q$ étant premier, dans lequel il n'est pas de Carmichael et dont le discriminant est premier avec $n$. Nous reprenons donc l'algorithme \ref{test-primalite-korselt} et l'utilisons avec les paramètres suivants.

\begin{table}[H]\label{param-korselt-cyclo}
	\begin{center}
		\begin{tabularx}{\textwidth}{|Y|Y|}
			\hline
			paramètre & valeur \\
			\hline
			\hline
			corps testés & corps cyclotomiques de la forme $\Q(\zeta_q)$ où $q$ est un nombre premier dans $[\![3, 300]\!]$\footnote{Il y a 62 tels nombres premiers.} \\\hline
			entiers de Carmichael testés & tous ceux de la liste \ref{liste-carmichael} \\\hline
		\end{tabularx}
		\caption{Paramètres des simulations du critère de Korselt pour les corps cyclotomiques.}
	\end{center}
\end{table}

Avec ces paramètres, le critère de Korselt a donné pour chaque élément $n$ de la liste, plusieurs corps cyclotomiques de discriminant premier avec $n$ dans lesquels $n$ n'est pas de Carmichael. Pour l'entier $n = 561$, nous trouvons par exemple qu'il est de Carmichael dans $\Q(\zeta_3)$ mais dans aucun autre corps testé. Cet algorithme permet aussi de prouver que l'entier de Howe \ref{Howe} est composé~! Nous trouvons qu'il est de Carmichael dans $\Q(\zeta_3)$ mais dans aucun autre corps testé\footnote{Nous sommes en réalité allés plus loin pour l'entier de Howe et avons testé tous les corps cyclotomiques de la forme $\Q(\zeta_q)$ où $q$ est un nombre premier compris entre $3$ et $600$.} En fait, nous avons trouvé très peu de couples $(n, K)$ où $n$ est un entier de Carmichael de la liste et $K$ un corps cyclotomique dans lequel $n$ est de Carmichael. La liste exhaustive de tels couples est la suivante.

\begin{table}[H]
	\begin{center}
		\begin{tabular}{|c|c|}
			\hline
			entier de Carmichael & corps cyclotomique dans lequel il est de Carmichael \\
			\hline
			\hline
			561				& $\Q(\zeta_3)$ \\\hline
			2821			& $\Q(\zeta_3)$ \\\hline
			1729			& $\Q(\zeta_3)$ \\\hline
			8911			& $\Q(\zeta_3)$ \\\hline
			15841			& $\Q(\zeta_3)$ \\\hline
			29341			& $\Q(\zeta_3)$ \\\hline
			46657			& $\Q(\zeta_3)$ \\\hline
			52633			& $\Q(\zeta_3)$ \\\hline
			63973			& $\Q(\zeta_3)$ \\\hline
			115921			& $\Q(\zeta_3)$ \\\hline
			126217			& $\Q(\zeta_3)$ \\\hline
			172081			& $\Q(\zeta_3)$ \\\hline
			115921			& $\Q(\zeta_3)$ \\\hline
			188461			& $\Q(\zeta_3)$ \\\hline
			252601			& $\Q(\zeta_5)$ (attention, changement) \\\hline
			294409			& $\Q(\zeta_3)$ \\\hline
			488881			& $\Q(\zeta_3)$ \\\hline
			512461			& $\Q(\zeta_3)$ et $\Q(\zeta_5)$ \\\hline
			entier de Howe	& $\Q(\zeta_3)$ \\\hline
		\end{tabular}
	\end{center}
	\caption{Liste exhaustive des couples trouvés $(n, K)$ où $n$ est un entier de Carmichael et $K$ un corps cyclotomique dans lequel $n$ reste de Carmichael, pour les paramètres \ref{param-korselt-cyclo}.}
\end{table}

Voici les faits remarquables à retenir de ces simulations.
	\begin{itemize}
		\item Nous n'avons trouvé aucun corps cyclotomique parmi ceux testés dans lequel les entiers suivants sont de Carmichael : 561, 1105, 2465, 6601, 1085, 41041, 62745, 101101, 449065.
		\item Les autres entiers sont de Carmichael dans $\Q(\zeta_3)$ ou $\Q(\zeta_5)$, mais dans aucun autre corps testé.
		\item Les deux seuls entiers de Carmichael de la liste de Carmichael dans $\Q(\zeta_5)$ sont 252601 et 512461. Leurs facteurs premiers sont congrus à $1$ modulo $5$. À noter que nous avons exhibé des entiers qui ne sont pas de Carmichael, dont tous les facteurs premiers ne sont pas congrus à $1$ modulo $5$ et qui sont de Carmichael dans $\Q(\zeta_5)$. C'est par exemple le cas de $2047$ (qui n'est même pas congru à $1$ modulo $5$).. \NTS{lien vers liste complète}
		\item L'entier 512461 est l'unique entier testé qui soit de Carmichael dans deux corps cyclotomiques à la fois.
		\item L'entier de Howe est de Carmichael dans $\Q(\zeta_3)$ mais dans aucun autre corps testé.
	\end{itemize}

Donnons aussi quelques statistiques.

\begin{table}[H]
	\begin{center}
		\begin{tabular}{|c|c|}
			\hline
			nombre de corps testés & 1857 \\\hline
			nombre d'idéaux de Carmichael trouvés dans ces corps & 18 \\\hline
			proportion d'idéaux de Carmichael trouvés dans ces corps & 0,9 \% \\\hline
		\end{tabular}
		\caption{Statistiques des simulations du critère de Korselt sur les corps cyclotomiques pour les paramètres \ref{param-korselt-cyclo}.}
	\end{center}
\end{table}

Quant aux performances, les calculs sont cependant significativement plus longs dans le cas des corps cyclotomiques que dans le cas des corps quadratiques. On retrouve néanmoins le fait que lesdits temps restent sensiblement proches pour tout entier (y compris l'entier de Howe) et tout corps cyclotomique testé.

\begin{table}[H]
	\begin{center}
		\begin{tabular}{|c|c|}
			\hline
			corps cyclotomique & temps de calcul \\
			\hline
			\hline
			$\Q(\zeta_7)$ & 0.005 s \\\hline
			$\Q(\zeta_{101})$ & 1.626 s \\\hline
			$\Q(\zeta_{199})$ & 18.9 s \\\hline
			$\Q(\zeta_{293})$ & 38.4 s \\\hline
		\end{tabular}
		\caption{Temps de calcul du critère Korselt dans les corps cyclotomiques donnés pour l'entier $n=512461$.}
	\end{center}
\end{table}
