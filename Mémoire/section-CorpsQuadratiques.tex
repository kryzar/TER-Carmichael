\section{Corps quadratiques}

\subsection{Théorie}

Entrons dès à présent dans le vif du sujet. 

\begin{theoreme}[2.5 dans l'article]\label{theoreme-2.5}
	Soit $n$ un entier impair sans facteurs carrés. S'il existe un diviseur premier $p$ de $n$ vérifiant $$p^2 - 1 \nmid n^2 - 1,$$ il existe une infinité de corps quadratiques $K$ dans lesquels $n$ \emph{n'est pas} de Carmichael.
\end{theoreme}

\begin{MotSurPreuve}
	On présage dès l'énoncé la nature de la preuve : c'est le critère de Korselt généralisé \ref{korselt-generalise} et la caractérisation \ref{carac-korselt}. On commence par s'assurer que $n$ est sans facteurs carrés de sorte de contrôler la décomposition de $n\OK$ dans un corps quadratique $K$ donné. La partie la plus difficile de la preuve consiste à trouver les bons corps quadratiques. Elle est basée sur la connaissance d'un nombre premier $p$ comme dans les hypothèses du théorème et sur une savante utilisation du théorème chinois. Les techniques de base de la ramification permettent encore une fois de s'assurer que les corps construits vérifient bien ce qu'on leur demande de vérifier. \\
\end{MotSurPreuve}

Bien que cet énoncé ne semble pas optimal en pratique\footnote{Il n'y a à ce jour (3 juin 2020) pas d'algorithme efficace pour déterminer si un entier est sans facteurs carrés. Les algorithmes passent souvent par la décomposition en produit de facteurs premiers, ce qui ne nous arrange pas. \NTS{sourcer}.}, certains nombres de Carmichael vérifient ces hypothèses : c'est même le cas de tous les nombres de Carmichael de la liste \ref{liste-carmichael}. Il existe donc pour chacun d'eux une infinité de corps quadratiques dans lesquels ils ne sont pas de Carmichael, prouvant qu'ils sont composés, grâce à l'énoncé \ref{ptf-contraposee}. \\

Dans l'exemple 2.6 de l'article, l'auteur ouvre une voix intéressante : celle du test de Fermat dans les corps quadratiques. Il montre que $n = 561$ est composé en exhibant le corps quadratique $K = Q(\sqrt{13})$ puis le $K$-témoin de Fermat $\alpha = 2 + 1\cdot \left(\frac{1 + \sqrt{13}}{2}\right) \in \OK$, ce qui prouve la composition de $n$ d'après le théorème \ref{ptf-contraposee}. Bien qu'il ne donne pas les détails, cela laisse penser que plutôt que d'utiliser le critère de Korselt\footnote{Ce qui nécessiterait de décomposer l'idéal $561\mathcal{O}_{\Q(\sqrt{13})}$.}, ce dernier a directement cherché un entier algébrique $\alpha$ ne vérifiant pas la congruence \ref{congruence-Carmichael}. Cet algorithme est appelé \emph{test de Fermat dans les corps quadratiques} et est un critère de composition. Étant donné un entier $n$ dont on veut prouver la composition, une version simple du test est la suivante.

\vspace{1em}
\begin{algorithm}[H]\label{test-Fermat-quadratique}
\caption{Test de Fermat dans un corps quadratique}
\KwIn{$n$ (entier à tester) \\
	$K$ (corps quadratique) \\ 
	$S_\alpha$ (ensemble de coordonnées pour $\alpha$)}
\If{$n$ et $\Disc(K)$ sont premiers entre eux}{
	$\theta\leftarrow$ un générateur de $\OK$ \\
	\ForEach{$\alpha = x+ y\theta$, $x, y\in S_\alpha$}{
		\If{$\alpha^{n^2}\not\equiv \alpha \pmod{n\OK}$}{
			\KwRet{$n$ n'est pas de Carmichael dans $K$ et est composé}	\\
			\textbf{arrêter le programme}
		}
	}
}
\end{algorithm}
\vspace{1em}

Nous discutons plus en détails de cet algorithme dans la section suivante. Il peut être tentant d'utiliser cet algorithme pour déterminer la primalité de $n$ \emph{avec une certitude morale}, dans l'esprit test de primalité de Rabin-Miller (voir \cite{Demazure}, §3.3.7, p. 68). L'idée serait que si $n$ vérifie la congruence $\alpha^{\Norm(n\OK)} \equiv \alpha \pmod{n\OK}$ pour un nombre suffisamment grand de corps quadratiques $K = \Q(\sqrt{d})$ de discriminant premier avec $n$ et d'entiers algébriques $\alpha \in \OK$, nous aurions une certitude morale de la primalité de $n$. Nous allons cependant voir qu'un tel test ne saurait exister. Nous avons évoqué la réciproque du petit théorème de Fermat (\ref{ptf-reciproque}). Il faut bien faire attention au fait que l'on ne peut pas remplacer l'hypothèse \og $n \nmid \Disc(K)$\fg par l'hypothèse \og $n$ et $\Disc(K)$ sont premiers entre eux \fg \footnote{Cela peut être contre intuitif : si $n$ est un entier et $K$ un corps de nombres, le fait que $n$ soit de Carmichael dans $K$ implique que $n$ et $\Disc(K)$ sont premiers entre eux.}~! Le mathématicien E.W. Howe montre en effet le résultat suivant.

\begin{theoreme}[Howe, 2000]
	Il existe un entier $h$ qui soit à la fois \emph{composé} et de Carmichael dans tout corps quadratique de discriminant premier avec $h$.
\end{theoreme}

En guise de preuve, Howe exhibe un tel nombre, défini par
	\begin{equation}\label{Howe}
		h = 17 \cdot 31 \cdot 41 \cdot 43 \cdot 89 \cdot 97 \cdot 167 \cdot 331 = 443372888629441.
	\end{equation}

Nous l'appellerons dans la suite \emph{entier de Howe}. L'existence de cet entier signifie que — comme dans $\Q$ — le test de Fermat \ref{test-Fermat-quadratique} ne peut pas prouver qu'un entier est composé simplement parce qu'il n'a de $K$-témoins de Fermat dans aucun corps quadratique $K$ dont le discriminant lui est premier. Plus généralement, l'auteur de l'article montre le théorème suivant.

\begin{theoreme}[2.7 dans l'article]\label{theoreme-2.7}
	Soient $n$ un entier composé sans facteur carré et $d\geq 1$ un entier. Si pour tout diviseur $p$ de $n$ et tout enter $1\leq i \leq d$ la division $$p^i - 1\mid n^d - 1$$ est vérifiée, alors $n$ est de Carmichael dans $\Q$ et tout corps de nombres de degré $d$ de discriminant premier avec $n$.
\end{theoreme}

\begin{definition}
	Un entier $n$ vérifiant les hypothèses du théorème \ref{theoreme-2.7} pour un certain entier $d\geq 1$ est appelé \emph{nombre de Carmichael rigide d'ordre $d$}.
\end{definition}

L'entier de Howe \ref{Howe} est donc un nombre de Carmichael rigide d'ordre $2$. Ces résultats nous enseignent que les corps quadratiques ne peuvent pas fournir un test de Fermat fiable. 

\begin{corollaire}
	Soit $n$ un entier. Si $n$ est sans facteur carré et composé, alors les propositions suivantes sont équivalentes :
	\begin{itemize}
		\item l'entier $n$ est de Carmichael dans tout corps quadratique de discriminant premier avec $n$~;
		\item l'identité \[p^2 - 1 \mid n^2 - 1\] est vérifiée pour tout diviseur premier $p$ de $n$.
	\end{itemize}
\end{corollaire}

\begin{proof}
	C'est la conjonction des théorèmes \ref{theoreme-2.5} et \ref{theoreme-2.7}.
\end{proof}

Avant de passer aux simulations numériques, étudions brièvement les relations entre idéaux de Carmichael et extensions de corps.

\subsection{Interlude : idéaux de Carmichael et extensions de corps}

Dans toute cette sous-section, nous nous donnons $$\Q\subset K \subset L$$ une tour de corps de nombres et
\begin{itemize}
	\item $I\subset \OK$ un idéal de Carmichael de $K$ ~;
	\item $J \subset \OL$ un idéal de Carmichael de $L$.
\end{itemize}

\begin{question}
	L'idéal étendu $I\OL$ est-il de Carmichael dans $L$ et l'idéal restreint $J\cap\OK$ est-il de Carmichael dans $K$ ?
\end{question}

L'auteur de l'article a déjà répondu à la première partie de la question dans l'exemple 2.6 de l'article, en utilisant le test de Fermat dans les corps quadratiques.

\begin{proposition}\label{premier-ce}
	Il existe des tours de corps de nombres $\Q\subset K \subset L$ et des idéaux de Carmichael $I$ de $K$ pour lesquels l'idéal étendu $I\OL$ qui ne sont pas de Carmichael dans $L$.
\end{proposition}

\begin{proof}
	L'exemple 2.6 de l'article montre que l'entier de Carmichael $561$ n'est pas de Carmichael dans $\Q(\sqrt{13})$. On a alors l'énoncé en prenant $K = \Q$, $I = 561\OK$ et $L = \Q(\sqrt{13})$.
\end{proof}

L'auteur du présent mémoire répond à la deuxième partie de la question en implémentant le critère de Korselt \ref{test-primalite-korselt} dans les corps quadratiques et donnant des exemples de tels idéaux corps de nombres.

\begin{proposition}\label{deuxieme-ce}
	Il existe des tours de corps de nombres $\Q\subset K \subset L$ et des idéaux de Carmichael $J$ de $L$ pour lesquels l'idéal restreint $J\cap \OK$ n'est pas de Carmichael dans $K$.
\end{proposition}

\begin{proof}
Donnons plusieurs exemples. On peut prendre :
\begin{itemize}
	\item $K = \Q$, $L = \Q(\sqrt{11})$ et $J = 35\OL$~;
	\item $K = \Q$, $L = \Q(\sqrt{95})$ et $J = 8029\OL$.
\end{itemize}
\end{proof}

\NTS{lien vers résultats complets}

\begin{remarque}
L'entier $n = 8029$ a la particularité d'être le produit de trois nombres premiers distincts : \[n = 8029 = 7 \cdot 31 \cdot 37.\] Le fait qu'il existe un corps quadratique dans lequel $n$ n'est pas de Carmichael peut se traduire dans le cadre de l'arithmétique classique grâce au critère de Korselt : existe-t-il un tripler de nombres premiers distincts $p$, $q$ et $r$ pour lesquels \[(pqr)^2 - 1\] est divisible à la fois par $p-1$, $q-1$ et $r-1$ ? Les diviseurs premiers de $8029$ sont un tel triplet.
\end{remarque}

Face à la généralité de l'énoncé de la question posée, il était tentant d'aller chercher une réponse ou bien théorique, ou bien dans des corps de nombres beaucoup plus compliqués. Le fait d'avoir trouvé une réponse avec deux algorithmes naïfs (test de Fermatet critère de Korselt) est symptomatique du fait que nous sommes largement ignorants sur le comportement d'un idéal de Carmichael lorsqu'on le restreint ou l'étend, et ce même dans le cas de corps aussi simples que les corps quadratiques. Nous verrons dès la prochaine sous-section de nombreux exemples de nombres de Carmichael qui le restent ou ne le restent pas dans tels ou tels corps quadratiques puis dans tels ou tels corps cyclotomiques.

\subsection{Simulation}

\subsubsection{Test de Fermat}

Testons ici le test de Fermat dans les corps quadratiques, plus précisément, l'algorithme \ref{test-Fermat-quadratique}, que l'auteur du présent mémoire a implémenté. Nous le testons pour chaque entier de Carmichael de la liste \ref{liste-carmichael} et choisissons\footnote{L'entier $43$ est le premier nombre premier qui ne soit facteur d'aucun des éléments de la liste \ref{liste-carmichael}. Cela assure que chaque entier de ladite liste est premier avec le discriminant de $\Q(\sqrt{43})$.} le corps quadratique \[K = \Q(\sqrt{43})\] et l'ensemble de coordonnées \[S_\alpha = [\![-2, +2]\!].\] Résumons ici les résultats obtenus.
\begin{itemize}
	\item L'algorithme fournit des $K$-témoins de Fermat pour 561, 2465, 2821, 8911, 10585, 15841, 29341, 46657, 52633 et 62745.
	\item L'algorithme ne fournit aucun $K$-témoin de Fermat pour 1729, 6601 et 41041.
	\item Dès lors que l'algorithme a trouvé un $K$-témoin de Fermat pour un certain entier, il en a trouvé plusieurs. Pour l'entier $n = 10585$, l'algorithme fournit par exemple \[(-2, -2), \; (-2, -1), \; (-2, 1), \; (-1, -1), \; (-1, 1), \; (1, 1),\] où ces éléments sont donnés par leurs coordonnées dans la base canonique $(1, \sqrt{43})$ de $\OK$. Attention, $(1, 2)$ n'est pas un $K$-témoin de Fermat pour $n$.
\end{itemize}

En choisissant\footnote{Arbitrairement.} cette désormais le corps \[K = \Q(\sqrt{-7})\], l'algorithme fournit des $K$-témoins de Fermat pour $6601$ (qui n'en avait pas pour $\Q(\sqrt{43})$ mais pas pour $2425$ (qui en avait plusieurs pour $\Q(\sqrt{43})$. \\

L'algorithme souffre néanmoins d'un gros problème de performances. Si les calculs prennent une fraction de secondes pour les premiers entiers, il faut 5 min à l'algorithme pour terminer sur l'entier $15841$, 42 min pour $41041$ puis 2 h 48 pour 62745, avant de planter pour les entiers suivants. Cela s'explique par les calculs de puissances. Si $\alpha\in \OK$ est un $K$-témoin de Fermat potentiel, il faut calculer $\alpha^{n^2}$. Lorsque $n$ devient grand, le coût de calcul devient prohibitif, dépassant largement les capacités d'un ordinateur personnel standard. Si l'on pose $n = 10848$ et $\alpha = 1 + 1\cdot \sqrt{43}$, les coordonnées de $\alpha^{n^2}$ sont tout de même des nombres à plusieurs milliers de chiffres. Cet algorithme est donc inutilisable en l'état pour tester la primalité de grands nombres. Si l'on ajoute à cela le fait qu'il n'est pas capable de détecter tous les entiers composés — comme l'entier de Howe — (voir section précédente), son usage semble définitivement à proscrire.
	

\subsubsection{Critère de Korselt}

Toujours dans l'objectif de numériquement prouver que les entiers de la liste \ref{liste-carmichael} sont effectivement composés, nous employons cette fois le critère de composition de Korselt \label{test-primalite-korselt} dans des corps quadratiques. Pour tout entier $n$, nous cherchons une liste de corps quadratiques dans lesquels $n$ n'est pas de Carmichael. Pour cela, nous lançons l'algorithme avec les paramètres suivant.

\begin{figure}[h!]\label{param-korselt-quadra}
	\begin{center}
		\begin{tabular}{|c|c|}
			\hline
			corps testés & corps quadratiques de la forme $\Q(\sqrt{d})$ \\ & où $d$ est un entier sans facteur carré \\ & dans $[\![-5000, +5000]\!]$\footnote{Il y a 6084 tels entiers.} \\\hline
			entiers de Carmichael testés & tous ceux de la liste \ref{liste-carmichael} \% \\\hline
		\end{tabular}
		\caption{Paramètres des simulations du critère de Korselt pour les corps quadratiques.}
	\end{center}
\end{figure}

Pour ces paramètres, le critère de Korselt est capable de donner pour tout élément $n$ de la liste, plusieurs corps quadratiques de discriminant premier avec $n$ dans lesquels $n$ n'est pas de Carmichael. En fait, il en exhibe des centaines. Il est donc impossible d'énoncer ici tous les résultats, mais citons à titre d'exemple que $561$ n'est pas de Carmichael dans $\Q(\sqrt{-4874})$ mais qu'il l'est dans $\Q(\sqrt{4877})$ ou que $172081$ n'est pas de Carmichael dans $\Q(\sqrt{766})$ mais qu'il l'est dans $\Q(\sqrt{-1459})$. En fait, pour la plupart des couples $(n, K)$ testés, $n$ n'est pas de Carmichael dans $K$. Nous avons tiré les statistiques suivantes de nos simulations.

\begin{figure}[h!]
	\begin{center}
		\begin{tabular}{|c|c|}
			\hline
			nombre de corps testés & 143524 \\\hline
			nombre d'idéaux de Carmichael trouvés dans ces corps & 34138 \\\hline
			proportion d'idéaux de Carmichael trouvés dans ces corps & 0,23786 \% \\\hline
		\end{tabular}
		\caption{Statistiques des simulations du critère de Korselt sur les corps quadratiques pour les paramètres \ref{param-korselt-quadra}.}
	\end{center}
\end{figure}

Si l'on cherche donc à s'assurer de la composition d'un entier $n$ pour lequel il existe au moins un corps quadratique de discriminant premier avec $n$ dans lequel $n$ n'est pas de Carmichael, ces résultats semblent indiquer qu'on a de bonnes chances d'en trouver un rapidement. Bien sûr, ces sommaires statistiques ne prouvent rien et on ne sait pas déterminer a priori s'il existe un tel corps quadratique. \\

Quant aux performances, les calculs sont extrêmement rapides dans tous les corps et semblent homogènes. On a par exemple les données suivantes.

\begin{figure}[h!]
	\begin{center}
		\begin{tabular}{|c|c|}
			\hline
			corps cyclotomique & temps de calcul \\
			\hline
			\hline
			$\Q(\sqrt{-4957})$ & 0.00303 s \\\hline
			$\Q(\sqrt{-2426})$ & 0.00211 s \\\hline
			$\Q(\sqrt{-2})$ & 0.0024 s \\\hline
			$\Q(\sqrt{+2})$ & 0.0024 s \\\hline
			$\Q(\sqrt{+2426})$ & 0.0031 s \\\hline
			$\Q(\sqrt{+4957})$ & 0.0031 s \\\hline
		\end{tabular}
		\caption{Temps de calcul du critère Korselt dans les corps quadratiques donnés pour l'entier $n=512461$.}
	\end{center}
\end{figure}
