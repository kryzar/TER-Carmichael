\section{Bilan des simulations}

Dans toute cette sous-section, nous nous donnons $$\Q\subset K \subset L$$ une tour de corps de nombres et \begin{itemize}
	\item $I\subset \OK$ un idéal de Carmichael de $K$ ~;
	\item $J \subset \OL$ un idéal de Carmichael de $L$.
\end{itemize}

\begin{question}
	L'idéal étendu $I\OL$ est-il de Carmichael dans $L$ et l'idéal restreint $J\cap\OK$ est-il de Carmichael dans $K$ ?
\end{question}

Nous avons déjà montré la première partie de la question sans la sous-section précédente ~! En prenant $K=\Q$ et un corps quadratique pour $L$, nous avons établi que pour tout entier $n$ de Carmichael de la liste \ref{liste-carmichael}, il existait plusieurs corps quadratiques dans lesquels $n$ n'est pas de Carmichael. 

\begin{proposition}\label{premier-ce}
	Il existe des tours de corps de nombres $\Q\subset K \subset L$ et des idéaux de Carmichael $I$ de $K$ pour lesquels l'idéal étendu $I\OL$ qui ne sont pas de Carmichael dans $L$.
\end{proposition}

Intéressons nous à la deuxième partie de la question.

\begin{proposition}\label{deuxieme-ce}
	Il existe des tours de corps de nombres $\Q\subset K \subset L$ et des idéaux de Carmichael $J$ de $L$ pour lesquels l'idéal restreint $J\cap \OK$ n'est pas de Carmichael dans $K$.
\end{proposition}

Avant de se lancer dans cette étude, il convient d'aller explorer le royaume des corps cyclotomiques, ce qui s'avérera porteur d'espoir pour l'élaboration d'un test de primalité. 

Il n'aura pas fallu aller chercher bien loin pour trouver des exemples de tels corps et idéaux. Prendre $K = \Q$ et un corps quadratique pour $L$ aura suffi. Si l'on dispose d'un idéal de Carmichael dans un corps de nombres donné et d'une extension finie de ce corps de nombres, il s'avère difficile de déterminer si l'idéal reste de Carmichael dans le grand corps de nombres.

