\section{Délices de la théorie}

Fort heureusement, certaines propriétés fondamentales des \emph{entiers} de Carmichael restent vraies dans le cadre plus général des \emph{idéaux} de Carmichael. Tout d'abord, le petit théorème de Fermat se généralise aux corps de nombres galoisiens : dans une telle extension, un nombre premier engendre un idéal qui est soit premier, soit de Carmichael. Plus formellement, vient ceci.
\begin{theoreme}[petit théorème de Fermat généralisé, 2.3 dans l'article]
	Soient $p$ un nombre premier et $K$ un corps de nombre galoisien tel que $p\nmid \Disc(K)$. Alors, pour tout entier algébrique $\alpha \in \OK$, on a $$\alpha^{\Norm_{K/\Q}(p)} \equiv \alpha \pmod{p\OK}.$$
\end{theoreme}

Fait tout à fait remarquable, l'auteur de l'article fournit une réciproque au petit théorème de Fermat dans ce nouveau cadre des idéaux de Carmichael.

\begin{theoreme}[réciproque du petit théorème de Fermat généralisé, 2.3 dans l'article]
	Soit $n>2$ un entier composé. Alors il existe un corps \emph{quadratique} $K$ vérifiant $n\nmid \Disc(K)$ et un entier algébrique $\alpha \in \OK$ tels que $$\alpha^{\Norm_{K/\Q}(n)} \not\equiv \alpha \pmod{n\OK}.$$
\end{theoreme}

\begin{MotSurPreuve}
	À l'instar de beaucoup d'autres résultats de l'article, la preuve de cet énoncé jouit à la fois d'une complexité technique raisonnable et d'une grande ingéniosité. Connaissant un diviseur $p$ premier de $n$, l'auteur construit un corps quadratique $K = \Q\left(\sqrt{(-1)^\frac{p-1}{2} p}\right)$ dans lequel $n$ n'est pas de Carmichael. Ce dernier point se démontre avec des techniques de base de la ramification. \\
\end{MotSurPreuve}

Nous pouvons de plus mettre à bout ces deux résultats pour fournir cette délicieuse équivalence.

\begin{theoreme}[petit théorème de Fermat généralisé et sa réciproque]\label{ptf—reciproque}
	Soit $n>2$ un entier. Alors $n$ est premier \ssi pour tout corps \emph{quadratique} $K$ vérifiant $n\nmid \Disc(K)$ et tout entier algébrique $\alpha \in \OK$, on a $$\alpha^{\Norm_{K/\Q}(n)} \equiv \alpha \pmod{n\OK}.$$
\end{theoreme}

Au delà de sa force théorique, cette énoncé semble porter une valeur historique majeure. Le test de primalité de Fermat était le seul test de primalité à ne pas disposer d'une réciproque (pour le test d'Euler par exemple, c'est une équivalence). Cette absence de réciproque semblait bien être le prix à payer pour sa simplicité et son efficacité. Il aura certes fallu aller chercher la réciproque dans les corps de nombres, mais l'énoncé prouve que les corps quadratiques. Ces objets ne sont d'ailleurs pas si loin de l'arithmétique classique : Gauss les étudiait déjà. \\

Un autre résultat d'importance (démontré avant le petit théorème de Fermat généralisé dans l'article) est la généralisation du critère de Korselt \ref{korselt}. C'est ce résultat que l'on utilise en premier lieu pour déterminer si un entier est de Carmichael.

\begin{theoreme}[critère de Korselt généralisé, 2.2 dans l'article]\label{korselt-generalise}
	Soient $K$ un corps de nombres et $I$ un idéal de $\OK$. On prend garde à supposer que $I$ est composé. Alors $I$ est de Carmichael \ssi $I$ est sans facteurs carrés et pour tout idéal premier $\P$ divisant $I$, on a $$\Norm(\P) - 1 \mid \Norm(I) - 1.$$
\end{theoreme}

\begin{MotSurPreuve}
	La preuve du critère de Korselt généralisé est étonnamment proche de celle du critère de Korselt dans $\Q$. Une savante utilisation du théorème chinois permet de prouver que $I$ est sans facteurs carrés. L'auteur se sert en suite de cela pour montrer l'identité sur les normes.
\end{MotSurPreuve}

\begin{remarque}
	Le critère de Korselt que nous connaissons dans le cadre de l'arithmétique se déduit immédiatement du critère de Korselt généralisé en prenant $K=\Q$.
\end{remarque}

\begin{remarque}
	Il faut ici se montrer vigilant avec la nomenclature. Un idéal $I$ est de Carmichael dans un corps de nombres $K$ si $I$ est un idéal \textbf{composé} qui \emph{en plus de cela}, vérifie l'identité \ref{congruence-Carmichael} pour tout entier algébrique $\alpha \in \OK$. \\ 

\noindent Si l'on a un corps de nombres $L$ et un idéal $J$ de $\OK$, montrer que $\Norm(\P) - 1\mid \Norm(J) - 1$ pour tout idéal premier $\P$ de $\OL$ divisant $J$ ne suffit pas. La preuve du critère de Korselt généralisé \ref{korselt-generalise} nous enseigne que si $J$ est premier, $J$ vérifie également cette identité. Il faut donc indépendamment montrer que $J$ est composé si l'on veut montrer que $J$ est un idéal de Carmichael. Là est le cœur du problème. L'auteur de l'article fait lui-même une petite erreur en oubliant cette hypothèse dans l'énoncé du théorème 2.7 : il doit y supposer $n$ composé.
\end{remarque}

Avant de poursuivre, donnons un lemme qui permettra d'alléger les énoncés de l'article.

\begin{lemme}
	Soient $n$ un entier et $K$ un corps de nombres. Si $n$ est de Carmichael dans $K$, alors $n$ et $\Disc(K)$ sont premiers entre eux.
\end{lemme}

\begin{proof}
	Si $n$ et $\Disc(K)$ ne sont pas premiers entre eux, $n$ a un facteur premier qui se ramifie dans $\OK$. L'idéal $n\OK$ a donc un facteur carré, ce qui l'empêche d'être un idéal de Carmichael d'après le critère de Korselt généralisé \ref{korselt-generalise}.
\end{proof}

Ces résultats fournissent un début de théorie confortable, qui nous laisse envisager l'avenir avec espoir. Nous pouvons dès à présent nous confronter à une étude plus spécifique, celle des corps quadratiques.
