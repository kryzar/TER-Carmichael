\section{Délices de la théorie}

Certaines propriétés fondamentales des \emph{entiers} de Carmichael restent vraies dans le cadre plus général des \emph{idéaux} de Carmichael. Commençons par un lemme.

\begin{lemme}\label{identite-1}
	Soient $K$ un corps de nombres, $\p$ un idéal premier de $\OK$ et $\alpha \in \OK$ un entier. Alors la congruence \[\alpha^{\Norm(\p)} \equiv \alpha \pmod{\p}\] est vérifiée.
\end{lemme}

\begin{proof}
	Si $\alpha\equiv 0\pmod{\p}$, le résultat est évident. Si ce n'est pas le cas, l'idéal $\p$ étant un idéal premier d'un anneau de Dedekind, il est maximal. Le quotient $\OK/\p$ est donc un corps et $\Norm(\p) -1$ est le cardinal du groupe des inversibles de $\OK/\p$. Le théorème de Lagrange implique donc $\alpha^{\Norm(\p)-1}\equiv 1 \pmod{\p}$ puis \[\alpha^{\Norm(\p)} \equiv \alpha \pmod{\p}\] en multipliant par $\alpha$ des deux côtés.
\end{proof}

Nous pouvons dès à présent généraliser le critère de Korselt \ref{korselt}.

\begin{theoreme}[critère de Korselt généralisé, \cite{article} th. 2.2]\label{korselt-generalise}
	Soient $K$ un corps de nombres et $I$ un idéal (premier ou composé) de $\OK$. Les assertions suivantes sont équivalentes :
	\begin{itemize}
		\item pour tout entier $\alpha \in \OK$, la congruence \[\alpha^{\Norm(I)} \equiv \alpha \pmod{I}\] est vérifiée ~;
		\item l'idéal $I$ est sans facteurs carrés et chacun de ses facteurs premier $\p$ vérifie l'identité \[\Norm(\p) - 1 \mid \Norm(I) - 1.\]
	\end{itemize}
\end{theoreme}

\begin{proof}
	Commençons par le sens réciproque. Soit $\alpha\in \OK$ un entier et $\p$ un idéal premier de $\OK$ divisant $I$.
	\begin{itemize}
		\item Si $\alpha \notin \p$, vient $\alpha^{\Norm(\p) - 1} \equiv 1 \pmod{\p}$ d'après le lemme \ref{identite-1}. L'entier $\Norm(\p) - 1$ divisant $\Norm(I) - 1$ par hypothèse, cela entraîne $\alpha^{\Norm(I)-1}\equiv 1 \pmod{\p}$ et donc \[\alpha^{\Norm(I)} \equiv \alpha \pmod{\p}.\]
		\item Si $\alpha\in \p$, la dernière congruence est toujours vérifiée.
	\end{itemize}
	L'idéal $I$ étant de plus sans facteurs carrés (hypothèse), il est de la forme $I = \p_1 \cdots \p_r,$ où les $\p_i$ sont des idéaux premiers distincts de $\OK$. Ces idéaux sont même maximaux (anneau de Dedekind) et donc comaximaux. Le théorème chinois affirme alors que l'application canonique
\begin{center}
	\begin{tabular}{ccc}
		$\OK \big/I$ & $\longrightarrow$ & $\OK\big/\p_1\times \dots \times \OK\big/ \p_r$ \\
		$a\pmod{I} $	& $\longmapsto$ & $\left(a\pmod{\p_1} , \dots, a\pmod{\p_r}\right)$,
	\end{tabular}
\end{center}
est un isomorphisme d'anneaux. Comme nous avons montré $\alpha^{\Norm(I)} \equiv \alpha \pmod{\p_i}$ pour tout $1\leq i \leq r$, l'isomorphisme évoqué entraîne la congruence \[\alpha^{\Norm(I)} \equiv \alpha \pmod{I}.\]

	Démontrons cette fois le sens direct. La preuve se fait en deux temps. Écrivons $I = \p_1^{e_1} \cdot \dots \cdot \p_r^{e_r}$, les $\p_i$ étant des idéaux premiers de $\OK$ distincts. Pour tout indice $1\leqslant i \leqslant r$ donnons nous $\alpha_i\in \OK$ un élément dont la classe modulo $\p_i$ engendre le groupe quotient $(\OK/\p_i)^\times$. Un tel élément existe car $\OK/\p_i$ est un corps fini et que le groupe des inversibles d'un corps fini est cyclique. En particulier, $\alpha_i$ n'est \emph{pas} dans $\p_i$. On a alors \[\alpha_i^{\Norm(I) - 1} \equiv 1 \pmod{\p_i}\] en reprenant le raisonnement précédent. D'après le théorème de Lagrange, l'ordre de $\alpha_i$ modulo $\p_i$ divise $\Norm(I) - 1$. Cet ordre étant $\Norm(\p_i) - 1$, on en déduit la division désirée. 

	Démontrons désormais que $I$ est sans facteurs carrés. Supposons qu'il existe un idéal premier $\p$ de $\OK$ tel que $\p^2\mid I$ et posons \[H = \left(\OK\big/\p^2\right)^\times.\] On a \[\left|H\right| = \Norm(\p)(\Norm(\p) - 1).\] Soit $p\in \Z$ l'unique nombre premier tel que $p\Z = \p\cap \Z$. L'entier $\Norm(\p)$ est une puissance de $\p$, d'où $p\mid \Norm(\p)$ et \[p\mid \left| H \right|.\] Le théorème de Cauchy abélien assure alors qu'il existe un élément $\alpha \in H$ d'ordre $p$. Comme $I$ est de Carmichael par hypothèse, on a $\alpha^{\Norm(I)}\equiv \alpha \pmod{I}$ puis $\alpha^{\Norm(I)} \equiv \alpha \pmod{\p^2}$, car $\p^2\mid I$. Comme en outre $\alpha\notin \p^2$, on en déduit $\alpha^{\Norm(I) - 1} \equiv 1 \pmod{\p^2}$ puis \[p \mid \Norm(I) - 1\] d'après le théorème de Lagrange. Comme $p\mid \Norm(I)$, cela constitue une contradiction. L'idéal $I$ est donc sans facteurs carrés.
\end{proof}

\begin{remarque}
	Le critère de Korselt que nous connaissons dans le cadre de l'arithmétique se déduit immédiatement du critère de Korselt généralisé en prenant $K=\Q$.
\end{remarque}

On a alors la (fondamentale) caractérisation suivante.

\begin{corollaire}\label{carac-korselt}
	Soient $K$ un corps de nombres et $I$ un idéal de $\OK$. Alors l'idéal $I$ est de Carmichael \ssi il est composé et vérifie les hypothèses du critère de Korselt généralisé \ref{korselt-generalise}.
\end{corollaire}

\begin{remarque}
	Il faut être vigilant avec la nomenclature. Pour établir qu'un idéal est de Carmichael, il faut montrer qu'il est composé, indépendamment du critère de Korselt.
\end{remarque}

Ce critère permet de généraliser le petit théorème de Fermat. Dans une extension galoisienne $K$ de $Q$ de degré fini, un nombre premier est soit inerte, soit de Carmichael dans $K$. Plus précisément.

\begin{theoreme}[petit théorème de Fermat généralisé, \cite{article} th. 2.2]\label{ptf}
	Soient $p$ un nombre premier et $K/\Q$ une extension galoisienne de degré fini tels que $p\nmid \Disc(K)$. Alors, pour tout entier $\alpha \in \OK$, on a $$\alpha^{\Norm_{K/\Q}(p)} \equiv \alpha \pmod{p\OK}.$$
\end{theoreme}

\begin{proof}
	Comme $p\nmid \Disc(K)$, le nombre premier $p$ n'est pas ramifié dans $\OK$. Comme l'extension $K/\Q$ est galoisienne de degré fini, les indices de ramifications et degrés résiduels des idéaux de $\OK$ au dessus de $\p$ sont égaux. L'idéal $p\OK$ est donc de la forme \[p\OK = \p_1\cdot \dots \p_r.\] Soit donc $f$ ce degré résiduel. On a $rf = [K : \Q]$ et surtout que $f$ divise $[ K : \Q ]$. Ainsi, pour tout indice $1\leq i \leq r$, il vient \[ \Norm(\p_i) - 1 = p^f - 1 \mid p^{[K : \Q]} - 1 = \Norm(p\OK) - 1.\] L'idéal $\OK$ vérifie donc le critère de Korselt généralisé \ref{korselt-generalise}, d'où le résultat.
\end{proof}

Nous étendons alors la notion de témoin de Fermat.

\begin{definition}\label{def-K-temoin}
	Soient $n$ un entier rationnel, $K/\Q$ une extension galoisienne de degré fini et $\alpha\in \OK$ un entier. On dit que $\alpha$ est un $K$-témoin de Fermat pour $n$ si l'on a \[\alpha^{\Norm(n\OK)} \not \equiv \alpha \pmod{n\OK}.\]
\end{definition}

Un nombre premier n'a donc de témoins de Fermat dans aucun corps de nombres galoisien et un idéal de Carmichael est un idéal \textbf{composé} n'ayant aucun témoin de Fermat dans l'anneau d'entiers dans lequel il vit. Cette généralisation de la notion de témoin de Fermat apparaît naturellement dans la contraposée du théorème \ref{ptf}. Fait tout à fait remarquable, l'auteur de l'article donne une réciproque au petit théorème de Fermat généralisé \ref{ptf}.

\begin{theoreme}[réciproque du petit théorème de Fermat généralisé, \cite{article} th. 2.4]\label{ptf-reciproque}
	Soit $n>2$ un entier rationnel composé. Alors $n$ admet un témoin de Fermat dans les corps quadratique de la forme \[K = \Q\left(\sqrt{(-1)^{\frac{p-1}{2}}p}\right)\] où $p$ est un facteur premier impair de $n$.
\end{theoreme}

\begin{proof}
	Soit $K$ un corps quadratique comme dans l'énoncé. Le nombre premier $p$ se ramifie dans $K$ et $n\OK$ a un facteur carré. Il n'est donc pas de Carmichael d'après le critère de Korselt généralisé \ref{korselt-generalise} et la caractérisation \ref{carac-korselt}.
\end{proof}


Vient alors l'équivalence suivante.

\begin{theoreme}[petit théorème de Fermat généralisé et sa réciproque]\label{ptf-equivalence}
	Soit $n>2$ un entier rationnel. Alors $n$ est premier \ssi il n'a de $K$-témoin de Fermat dans aucun corps quadratique $K$ vérifiant $n\nmid \Disc(K)$.
\end{theoreme}

\begin{proof}
	Le sens direct correspond à \ref{ptf} et le sens réciproque s'obtient avec l'énoncé précédent en constatant que $\Disc(K) = p$ et $n\nmid p$.
\end{proof}

\begin{remarque}
Au delà de sa force théorique, cette énoncé semble porter une valeur historique notable. Le test de primalité de Fermat était le seul test de primalité classique à ne pas disposer d'une réciproque (pour le test d'Euler par exemple, c'est une équivalence). Cette absence de réciproque semblait bien être le prix à payer pour sa simplicité et son efficacité. Il aura certes fallu aller chercher la réciproque dans les corps de nombres, mais l'énoncé prouve que les corps quadratiques suffisent. Ces objets ne sont d'ailleurs pas si loin de l'arithmétique classique : Gauss les étudiait déjà.
\end{remarque}

Nous avons déjà des outils suffisamment robustes pour fournir un test de primalité naïf. Comme certains outils de calcul formel sont capables de décomposer un idéal de l'anneau d'entiers d'un corps de nombres en produit d'idéaux premiers dudit anneau : c'est notamment le cas de SageMath. Il est alors facile d'implémenter le critère de Korselt \ref{korselt-generalise}. La contraposée du petit théorème de Fermat généralisé \ref{ptf-equivalence} nous permet en suite d'écrire le critère de composition suivant.

\vspace{1em}
\begin{algorithm}[H]\label{test-primalite-korselt}
\caption{Critère de composition de Korselt dans les extensions galoisiennes de degré fini de $\Q$}
\KwIn{$n$ (entier rationnel à tester), $\mathscr{K}$ (liste d'extensions galoisiennes de degré fini de $\Q$)}
\ForEach{$K$ dans $\mathscr{K}$}{
	\If{$n \nmid \Disc(K)$}{
		\If{$n\OK$ ne vérifie pas le critère de Korselt}{
			\KwRet{$n\OK$ n'est pas de Carmichael et $n$ est composé} \\
			\textbf{arrêter le programme.}
		}
	}
}
\end{algorithm}
\vspace{1em}

Avant de poursuivre, donnons un lemme qui permettra d'alléger les énoncés de l'article.

\begin{lemme}
	Soient $n$ un entier rationnel et $K$ un corps de nombres. Si $n$ est de Carmichael dans $K$, alors $n$ et $\Disc(K)$ sont premiers entre eux.
\end{lemme}

\begin{proof}
	Si les entiers $n$ et $\Disc(K)$ ne sont pas premiers entre eux, $n$ a un facteur premier qui divise $\Disc(K)$ et qui se ramifie dans $\OK$. L'idéal $n\OK$ a donc un facteur carré, ce qui l'empêche d'être un idéal de Carmichael d'après le critère de Korselt généralisé \ref{korselt-generalise} et la caractérisation \ref{carac-korselt}.
\end{proof}

Ces résultats fournissent un début de théorie confortable. Nous pouvons dès à présent nous confronter à une étude plus spécifique, celle des corps quadratiques.
