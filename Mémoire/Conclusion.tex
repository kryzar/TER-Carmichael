\section*{Conclusion}
\addcontentsline{toc}{section}{Conclusion}

L'article de G.A. Steele propose une extension de la notion d'entier de Carmichael. Nous y retrouvons le petit théorème de Fermat, le critère de Korselt et exposons de précieuses clés pour la détection des entiers de Carmichael. Le théorème, \ref{theoreme-2.5} tout d'abord, force à faire la distinction entre les nombres de Carmichael et les nombres de Carmichael \emph{rigides} comme l'entier de Howe. Ce dernier indique toutefois que, comme $\Q$, les corps quadratiques ne peuvent pas détecter tous les entiers de Carmichael à l'aide du théorème de Fermat. C'est en revanche le cas des corps cyclotomiques : pour tout nombre de Carmichael $n$, il existe une infinité de corps cyclotomiques, de discriminant premier avec $n$, dans lesquels $n$ n'est pas de Carmichael ~; c'est le théorème \ref{theoreme-3.6}. Ces énoncés d'existence n'indiquent cependant pas comment trouver de tels corps ou des témoins de Fermat. \\

Nos simulations numériques ont mis en valeur des tendances allant dans le sens de ces résultats. Les entiers de Carmichael (dans $\Q$) testés restèrent de Carmichael dans environ 24 \% des corps quadratiques (\ref{statistiques-quadra}) contre moins de 1 \% des corps cyclotomiques (\ref{statistiques-cyclo}). Les versions cyclotomiques du test de Fermat et du critère de Korselt sont deux alternatives vraisemblables pour aboutir à un critère de composition. Le test de Fermat évite d'avoir à décomposer un idéal et le critère de Korselt évite de calculer des témoins de Fermat. Une étude sur la répartition de ces derniers pourrait aider à trancher. \\

Enfin, les résultats de la section 3 témoignent de notre ignorance actuelle quant au comportement des idéaux de Carmichael et des extensions de corps dans lesquelles ces derniers conservent cette propriété. G.A. Steele a par ailleurs concentré son étude sur des corps bien connus : les corps quadratiques et cyclotomiques. Il y utilise à volonté des outils de théorie algébrique des nombres : ramification, discriminant, et de théorie analytique des nombres : lemmes 3.3 et 3.4 de l'article, théorème de Heath-Brown. Le critère de Korselt est le seul outil propre à la théorie. L'expansion de celle-ci dans une plus grande généralité, pour lui fournir des outils souples et adaptés, semble cruciale son développement.
