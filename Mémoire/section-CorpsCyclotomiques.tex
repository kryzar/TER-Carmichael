\section{Corps cyclotomiques}

\subsection{Deux théorèmes}

L'étude des idéaux de Carmichael dans les corps cyclotomiques est porteuse d'espoir et fournit de beaux résultats susceptibles d'être à la base de tests de primalité, notamment le théorème 3.6. Notons que les seuls corps qui nous intéressent ici sont les corps cyclotomiques engendrés par des racines primitive $p$-ième de l'unité, où $p$ est premier. Commençons par un résultat théorique.

\begin{theoreme}[3.1 dans l'article]
	Pour tout entier naturel $n$ composé, il existe une infinité de corps de nombres \emph{abéliens} $K$ de discriminant premier avec $n$ dans lesquels $n$ \emph{n'est pas de Carmichael}.
\end{theoreme}

\begin{MotSurPreuve}
	Les preuves des énoncés sur les corps cyclotomiques sont autrement plus sophistiquées que celles sur les corps quadratiques et l'auteur y utilise à volonté des résultats de théorie analytique des nombres. Ces derniers assurent de l'existence d'objets mais ne les construisent pas, à l'image des lemmes 3.3 et 3.4 de l'article. Pour ce théorème, la construction des corps se fait avec la correspondance de Galois et la vérification et la vérification qu'ils vérifient les bonnes propriétés nécessite des arguments sophistiqués de ramification (groupes de décomposition et d'inertie, Frobenius d'un élément). Nous verrons ces arguments plus en détails dans la preuve du théorème 3.6 de l'article (n°\ref{theoreme-3.6} chez nous). \\
\end{MotSurPreuve}

Un nombre de Carmichael étant composé, il vérifie les hypothèses du théorème. Cela fournit une nouvelle réciproque au petit théorème de Fermat, plus contraignante que la précédente.

\begin{theoreme}[deuxième réciproque]
	Soit $n$ un entier. Alors $n$ est premier \ssi pour tout corps de nombres \emph{abélien} $K$ de discriminant premier avec $n$ et tout entier algébrique $\alpha \in \OK$, on a $$\alpha^{\Norm_{K/\Q}(n)} \equiv \alpha \pmod{n\OK}.$$
\end{theoreme}

Le résultat le plus à même d'aboutir à un test de primalité est le crucial théorème suivant. Il nous enseigne qu'il faut aller chercher du côté des corps cyclotomiques. Nous en donnons la démonstration.

\begin{theoreme}[3.6 dans l'article]\label{theoreme-3.6}
	Soit $n$ un entier composé ayant au moins trois facteurs premiers distincts. Alors il existe une infinité de corps cyclotomiques $K$ de la forme $K = \Q(\zeta_q)$, $q$ étant premier, tels que $\Disc(K)$ est premier avec $n$ et $n$ \emph{n'est pas de Carmichael dans $K$}.
\end{theoreme}

\begin{proof}
	\NTS{placeholder}
\end{proof}

Un nombre de Carmichael ayant toujours au moins trois diviseurs premiers distincts (voir \cite{Demazure}, Proposition 3.35, p. 90), il vérifiera toujours les hypothèses du théorème. Ce résultat est donc en théorie bien plus puissant que le théorème \ref{theoreme-2.5}, puisque nous avons vu que le test de primalité naïf qui en découlait ne détectait pas tous les entiers de Carmichael, comme par exemple l'entier de Howe \ref{Howe}. Nous allons donc essayer de construire un test de primalité avec ce théorème.

\subsection{Une application}

Commençons par un très joli corollaire, qui lui aussi montre la supériorité théorique des corps cyclotomiques par rapport aux corps quadratiques pour détecter des corps de nombres dans lesquels un entier de Carmichael n'est pas de Carmichael.

\begin{corollaire}[3.7 dans l'article]\label{corollaire-3.7}
	Soit $n$ un entier composé. Il existe au moins un corps cyclotomique de la forme $\Q(\zeta_q)$, $q$ étant premier, de discriminant premier avec $n$ dans lequel $n$ n'est pas de Carmichael.
\end{corollaire}

\begin{proof}
	Distinguons trois cas.
	\begin{itemize}
		\item Si $n$ a un unique facteur premier, étant composé, il a un facteur carré et ne sera jamais de Carmichael, d'après le critère de Korselt généralisé \ref{korselt-generalise}.
		\item Si $n$ a deux uniques facteurs premiers distincts, il n'est pas de Carmichael dans $\Q$, le corps cyclotomique $\Q(\zeta_2)$ (voir \cite{Demazure}, Proposition 3.35, p. 90).
		\item Si $n$ a au moins trois facteurs premiers distincts, nous appliquons le théorème précédent.
	\end{itemize}
\end{proof}

Ce corollaire a bien entendu droit à sa réciproque du théorème de Fermat.

\begin{theoreme}[troisième réciproque]
	Soit $n$ un entier. Alors $n$ est premier \ssi pour tout corps cyclotomique $K$ de la forme $K = \Q(\zeta_q)$, $q$ étant premier, et tout entier algébrique $\alpha\in \OK$, on a $$\alpha^{\Norm_{K/\Q}(n)} \equiv \alpha \pmod{n\OK}.$$
\end{theoreme}

\begin{remarque}
	Dans les deux réciproques que nous venons de donner, l'hypothèse sur le discriminants est que $n$ et $K$ doivent être premiers entre eux. C'est une hypothèse bien plus forte que de demander à ce que $n$ ne divise pas le discriminant de $K$ comme dans les corps quadratiques (\ref{ptf-reciproque}). Cela permet de tester beaucoup moins de corps.
\end{remarque}

Mais revenons au théorème \ref{theoreme-3.6}, duquel nous voulons tirer un critère de primalité. Étant donné un entier de Carmichael $n$, nous proposons l'algorithme suivant.

\vspace{1em}
\begin{algorithm}[H]
\KwIn{borne\_q}
\ForEach{$q$ nombre premier dans $[\![3, \mathrm{borne\_q}]\!]$}{
	$K \leftarrow \Q(\zeta_q)$ \;
	\If{$\pgcd(q, n) = 1$}{
		\If{$n$ n'est pas de Carmichael dans $K$}{
			\KwRet{$n$ n'est pas de Carmichael dans $\Q(\zeta_q)$ et est composé}	\\
			\textbf{fin du programme}
		}
	}
}
\end{algorithm}
\vspace{1em}

\begin{remarque}
	Pour tester si un nombre est de Carmichael dans un corps de nombres de donné, nous implémentons le critère de Korselt dans une fonction dédiée. Pour plus de détails sur l'implémentation de ces algorithmes, nous invitons le lecteur à se référer à l'annexe \ref{annexe-ce}.
\end{remarque}

Pour chacun des nombres $n$ de la liste $\mathfrak{C}$, cet algorithme a pu exhiber de nombreux corps cyclotomiques dans lesquels $n$ n'est pas de Carmichael, prouvant que $n$ est composé~! Par exemple,
\begin{itemize}
	\item $561$ n'est pas de Carmichael dans $\Q(\zeta_5)$ ; 
	\item $1729$ n'est pas de Carmichael dans $\Q(\zeta_{17})$ ;
	\item $512461$ n'est pas de Carmichael dans $\Q(\zeta_{83})$.
\end{itemize}

Nombre d'autres résultats sont disponibles sur la page GitHub de l'auteur : \url{https://github.com/kryzar/TER-Carmichael/blob/master/Scripts/Results_Corollary_3-7.txt}. \\

Cet algorithme permet aussi de prouver que l'entier de Howe est composé ! Notons $h$ cet entier. Mentionné après le théorème 2.7 de l'article, $h$ vaut $$h = 17 \cdot 31 \cdot 41 \cdot 43 \cdot 89 \cdot 97 \cdot 167 \cdot 331$$ et est de Carmichael non seulement dans $\Q$, mais aussi dans tout corps quadratique dont le discriminant est premier avec $h$ (on dit que $h$ est un nombre de Carmichael \emph{rigide d'ordre 2}). Notre algorithme exhibe toutefois de nombreux corps cyclotomiques dans lesquels $h$ n'est pas de Carmichael, comme $\Q(\zeta_{199})$. La liste complète des résultats trouvés est cette fois disponible à \url{https://github.com/kryzar/TER-Carmichael/blob/master/Scripts/Results_Howe_cyclotomic.txt}.
